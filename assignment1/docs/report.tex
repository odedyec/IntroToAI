\documentclass{article}                     % onecolumn (standard format)

\usepackage{graphicx}
\usepackage{float}
\usepackage{amsmath}
\usepackage{cite}
\usepackage{subfig}
\usepackage{rotating}
\usepackage[left=2.2cm, right=2.2cm]{geometry}
\usepackage{algorithm}
%\floatstyle{ruled}
\usepackage{hyperref}
\usepackage{algpseudocode}

\newfloat{algorithm}{tbp}{loa}
\providecommand{\algorithmname}{Algorithm}
\floatname{algorithm}{\protect\algorithmname}
%
% \usepackage{mathptmx}      % use Times fonts if available on your TeX system
%
% insert here the call for the packages your document requires
%\usepackage{latexsym}
% etc.
%
% please place your own definitions here and don't use \def but
% \newcommand{}{}
%

%
\begin{document}
	
\title{Introduction to AI - assignment 1}


\author{Oded~Yechiel         \and
	Matan~Rusanovsky
}

\date{13/11/18}

\maketitle

\section{Introduction}
	In this assignment we had to provide a hurricane simulator and 6 agents to roam the simulator and rescue people. The rescue is performed by loading the people from towns to shelter towns.
	
	Roads are connecting towns and are weighted with values. The vehicle traverse time between towns is calculated using the following formula, 
	\begin{equation}\label{eq:traverse_time}
	traverse\_time = W \times (1 + K \times P), 
	\end{equation}
	where, $ W $ is the weight between the towns, $ K $ is a slowdown parameter that is multiplied by the number of people in the vehicle.
	
	\subsection{Agents}
	The first three agents are Human agent, mainly for debugging purposes, a greedy agent, also for debugging purposes, and a Vandal agent, used to complicate the given problem.
	
	The last three agents are Greedy agents, based on Heuristics which will be explained in the next section, an A* agent and a real-time A* agent, which are based on the same Heuristics.

	\subsection{Performance}
	In order to assess the agents, each of the later agents will record the number of expansions they perform. 
	The overall performance, $ P $ is calculated as,
	\begin{equation}\label{eq:performance}
	P = f * S + N, 
	\end{equation}
	where, $ S $ is the number of people saved, and $ N $, is the number of expands the agent performs, and $ f < 0 $ is a factor to multiply $ S $. Usually $ | f | $ will be a large number.
	Lower performance value, $ P $, indicates better performance.
	\\
	\\
	
	
	In this report we shall explain the Heuristics used in this assignment. Please refer to our git repository for simulations and results:
	\url{https://github.com/odedyec/IntroToAI}



\section{Heuristics}
The chosen heuristics attempts to assess the number of people that there is no way to save. Since the performance is mainly based on the amount of people we save, each unsaved person is multiplied by a large value in the heuristic.

The calculation of the Heuristics can be computationally intense, since we need to find optimal roads to towns with people and to towns with shelter. Therefore, in order to keep these calculation to a minimum, we find shortest paths for each of the vertices that either have people, or shelter vertices. The closest shelter vertex is considered as the only shelter, for heuristic computation purposes. Then for each vertex with people the distance calculated is the shortest path to that vertex and back and to the shelter. The time to perform this distance is compared with the deadline to decide if it is possible to save those people. The heuristics is summarized below:

\begin{algorithm}
	\caption{Heuristic calculation}
	\begin{enumerate}
		\item calculate the path to all shelters
		\item find the closest shelter, $ d_s $
		\item find paths to all vertices with people
		\item $ time = Time[sim] $
		\item for each distance, $ d $, to vertex with $ P $ people, do
		\item $ time = time + d + (1 + K * P)*(d + d_s)$
		\item if $ time < deadline $ return number of people currently unsaved
		
	\end{enumerate}
	
\end{algorithm}

\end{document}
