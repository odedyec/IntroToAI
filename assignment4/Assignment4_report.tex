\documentclass{article}                     % onecolumn (standard format)

\usepackage{graphicx}
\usepackage{float}
\usepackage{amsmath}
\usepackage{cite}
\usepackage{subfig}
\usepackage{rotating}
\usepackage[left=2.2cm, right=2.2cm]{geometry}
\usepackage{algpseudocode}
\usepackage{algorithm}
%\floatstyle{ruled}
\usepackage{hyperref}
\usepackage{algpseudocode}

\newfloat{algorithm}{tbp}{loa}
\providecommand{\algorithmname}{Algorithm}
\floatname{algorithm}{\protect\algorithmname}
%
% \usepackage{mathptmx}      % use Times fonts if available on your TeX system
%
% insert here the call for the packages your document requires
%\usepackage{latexsym}
% etc.
%
% please place your own definitions here and don't use \def but
% \newcommand{}{}
%

%
\begin{document}
	
\title{Introduction to AI - assignment 4}


\author{Oded~Yechiel         \and
	Matan~Rusanovsky
}

\date{3/1/19}

\maketitle

\section{Introduction}
	In this assignment we provide a Bayesian Network for the hurricane problem from assignment 1.

\section{Bayesian Network construction}
We built our BN in the following manner. The nodes of the BN are as described in the assignment: \\
3 types of variables (BN nodes): \textbf{blockages} (one for each edge) \textbf{flooding} (one for each vertex,) and \textbf{evacuees} present (one for each vertex). \\
Each \textbf{flooding} node of vertex $v$ point to all \textbf{blockage} nodes of edges $\{\{v,u\} : u \in V\}$. \\
All \textbf{blockage} nodes of edges $\{\{v,u\} : v,u \in V\}$ point to \textbf{evacuees} nodes of vertices $u$ and $v$.

\section{Probabilistic reasoning algorithm}
Our probability reasoning algorithm is the simple \textit{Enumeration Algorithm}. 

\section{Run example}
\subsection{First example - the setting that was given in the assignment}
In this example we check that the following path is free of blocked edges: 1,2
\subsubsection{Input}
\begin{verbatim}
file: "input_graph2.txt"


#V 4          ; number of vertices n in graph (from 1 to n)

#V 0 F 0.2    ; Vertex 0, probability flooding 0.2
#V 1 F 0.4    ; Vertex 1, probability flooding 0.4


#E0 0 1 1 ; Edge0 between vertices 0 and 1, weight 1
#E1 1 2 3 ; Edge1 between vertices 1 and 2, weight 3
#E2 2 3 3 ; Edge2 between vertices 2 and 3, weight 3
#E3 1 3 4 ; Edge3 between vertices 1 and 3, weight 4
\end{verbatim}

\subsubsection{Output}

\begin{verbatim}
Vertex0
---------
P(Flood)=20.0%
P(!Flood)=80.0%
P(Evacuees|!Blockage0 )=0.1%
P(Evacuees|Blockage0 )=40.0%
P(!Evacuees|!Blockage0 )=99.9%
P(!Evacuees|Blockage0 )=60.0%

P(Evacuees 0 | []) = 0.13334032

Vertex1
---------
P(Flood)=40.0%
P(!Flood)=60.0%
P(Evacuees|!Blockage0 !Blockage1 !Blockage3 )=0.1%
P(Evacuees|!Blockage0 !Blockage1 Blockage3 )=40.0%
P(Evacuees|!Blockage0 Blockage1 !Blockage3 )=40.0%
P(Evacuees|!Blockage0 Blockage1 Blockage3 )=16.000000000000004%
P(Evacuees|Blockage0 !Blockage1 !Blockage3 )=40.0%
P(Evacuees|Blockage0 !Blockage1 Blockage3 )=16.000000000000004%
P(Evacuees|Blockage0 Blockage1 !Blockage3 )=16.000000000000004%
P(Evacuees|Blockage0 Blockage1 Blockage3 )=6.400000000000001%
P(!Evacuees|!Blockage0 !Blockage1 !Blockage3 )=99.9%
P(!Evacuees|!Blockage0 !Blockage1 Blockage3 )=60.0%
P(!Evacuees|!Blockage0 Blockage1 !Blockage3 )=60.0%
P(!Evacuees|!Blockage0 Blockage1 Blockage3 )=84.0%
P(!Evacuees|Blockage0 !Blockage1 !Blockage3 )=60.0%
P(!Evacuees|Blockage0 !Blockage1 Blockage3 )=84.0%
P(!Evacuees|Blockage0 Blockage1 !Blockage3 )=84.0%
P(!Evacuees|Blockage0 Blockage1 Blockage3 )=93.6%

P(Evacuees 1 | []) = 0.13003669191184003

Vertex2
---------
P(Flood)=0.0%
P(!Flood)=100.0%
P(Evacuees|!Blockage1 !Blockage2 )=0.1%
P(Evacuees|!Blockage1 Blockage2 )=40.0%
P(Evacuees|Blockage1 !Blockage2 )=40.0%
P(Evacuees|Blockage1 Blockage2 )=16.000000000000004%
P(!Evacuees|!Blockage1 !Blockage2 )=99.9%
P(!Evacuees|!Blockage1 Blockage2 )=60.0%
P(!Evacuees|Blockage1 !Blockage2 )=60.0%
P(!Evacuees|Blockage1 Blockage2 )=84.0%

P(Evacuees 2 | []) = 0.033506896599999995

Vertex3
---------
P(Flood)=0.0%
P(!Flood)=100.0%
P(Evacuees|!Blockage2 !Blockage3 )=0.1%
P(Evacuees|!Blockage2 Blockage3 )=40.0%
P(Evacuees|Blockage2 !Blockage3 )=40.0%
P(Evacuees|Blockage2 Blockage3 )=16.000000000000004%
P(!Evacuees|!Blockage2 !Blockage3 )=99.9%
P(!Evacuees|!Blockage2 Blockage3 )=60.0%
P(!Evacuees|Blockage2 !Blockage3 )=60.0%
P(!Evacuees|Blockage2 Blockage3 )=84.0%

P(Evacuees 3 | []) = 0.0255396766

Edge0
----------
P(Blockage 0|!flood0 !flood1)=0.001
P(Blockage 0|!flood0 flood1)=0.6
P(Blockage 0|flood0 !flood1)=0.6
P(Blockage 0|flood0 flood1)=0.84

P(Blockage 0 | []) = 0.33168

Edge1
----------
P(Blockage 1|!flood1 !flood2)=0.001
P(Blockage 1|!flood1 flood2)=0.19999999999999998
P(Blockage 1|flood1 !flood2)=0.19999999999999998
P(Blockage 1|flood1 flood2)=0.3599999999999999

P(Blockage 1 | []) = 0.0806

Edge2
----------
P(Blockage 2|!flood2 !flood3)=0.001
P(Blockage 2|!flood2 flood3)=0.19999999999999998
P(Blockage 2|flood2 !flood3)=0.19999999999999998
P(Blockage 2|flood2 flood3)=0.3599999999999999

P(Blockage 2 | []) = 0.0010000000000000002

Edge3
----------
P(Blockage 3|!flood1 !flood3)=0.001
P(Blockage 3|!flood1 flood3)=0.15
P(Blockage 3|flood1 !flood3)=0.15
P(Blockage 3|flood1 flood3)=0.2775000000000001

P(Blockage 3 | []) = 0.06059999999999999


------------
The probability that the given path is free from blockages is 0.9184806000000001
------------
\end{verbatim}

\subsubsection{Brief explanation}
As we can see, since the probability of the flooding on the vertices of the path: $1,2,3$ is low, the probability that the given path is free of blockages is high.

\subsection{Second example}
In this example we check that the following path is free of blocked edges: 0,2
\subsubsection{Input}
\begin{verbatim}
file: "input_graph1.txt"


#V 4          ; number of vertices n in graph (from 1 to n)

#V 0 F 0.8    ; Vertex 0, probability flooding 0.8
#V 1 F 0.5    ; Vertex 1, probability flooding 0.5
#V 2 F 0.5    ; Vertex 2, probability flooding 0.5
#V 3 F 0.2    ; Vertex 3, probability flooding 0.2


#E0 0 1 1 ; Edge0 between vertices 0 and 1, weight 1
#E1 0 2 1 ; Edge1 between vertices 0 and 2, weight 3
#E2 2 3 1 ; Edge2 between vertices 2 and 3, weight 3
\end{verbatim}

\subsubsection{Output}
\begin{verbatim}
Vertex0
---------
P(Flood)=80.0%
P(!Flood)=19.999999999999996%
P(Evacuees|!Blockage0 !Blockage1 )=0.1%
P(Evacuees|!Blockage0 Blockage1 )=40.0%
P(Evacuees|Blockage0 !Blockage1 )=40.0%
P(Evacuees|Blockage0 Blockage1 )=16.000000000000004%
P(!Evacuees|!Blockage0 !Blockage1 )=99.9%
P(!Evacuees|!Blockage0 Blockage1 )=60.0%
P(!Evacuees|Blockage0 !Blockage1 )=60.0%
P(!Evacuees|Blockage0 Blockage1 )=84.0%

P(Evacuees 0 | []) = 0.23206134805000003

Vertex1
---------
P(Flood)=50.0%
P(!Flood)=50.0%
P(Evacuees|!Blockage0 )=0.1%
P(Evacuees|Blockage0 )=40.0%
P(!Evacuees|!Blockage0 )=99.9%
P(!Evacuees|Blockage0 )=60.0%

P(Evacuees 1 | []) = 0.2548039

Vertex2
---------
P(Flood)=50.0%
P(!Flood)=50.0%
P(Evacuees|!Blockage1 !Blockage2 )=0.1%
P(Evacuees|!Blockage1 Blockage2 )=40.0%
P(Evacuees|Blockage1 !Blockage2 )=40.0%
P(Evacuees|Blockage1 Blockage2 )=16.000000000000004%
P(!Evacuees|!Blockage1 !Blockage2 )=99.9%
P(!Evacuees|!Blockage1 Blockage2 )=60.0%
P(!Evacuees|Blockage1 !Blockage2 )=60.0%
P(!Evacuees|Blockage1 Blockage2 )=84.0%

P(Evacuees 2 | []) = 0.22567338088000002

Vertex3
---------
P(Flood)=20.0%
P(!Flood)=80.0%
P(Evacuees|!Blockage2 )=0.1%
P(Evacuees|Blockage2 )=40.0%
P(!Evacuees|!Blockage2 )=99.9%
P(!Evacuees|Blockage2 )=60.0%

P(Evacuees 3 | []) = 0.15437559999999997

Edge0
----------
P(Blockage 0|!flood0 !flood1)=0.001
P(Blockage 0|!flood0 flood1)=0.6
P(Blockage 0|flood0 !flood1)=0.6
P(Blockage 0|flood0 flood1)=0.84

P(Blockage 0 | []) = 0.6361

Edge1
----------
P(Blockage 1|!flood0 !flood2)=0.001
P(Blockage 1|!flood0 flood2)=0.6
P(Blockage 1|flood0 !flood2)=0.6
P(Blockage 1|flood0 flood2)=0.84

P(Blockage 1 | []) = 0.6361

Edge2
----------
P(Blockage 2|!flood2 !flood3)=0.001
P(Blockage 2|!flood2 flood3)=0.6
P(Blockage 2|flood2 !flood3)=0.6
P(Blockage 2|flood2 flood3)=0.84

P(Blockage 2 | []) = 0.3844


------------
The probability that the given path is free from blockages is 0.22401684000000005
------------
\end{verbatim}

\subsubsection{Brief explanation}
As we can see, since the probability of the flooding on the vertices of the path: $0,2,3$ is high, the probability that the given path is free of blockages is low.

\section{How to run}
In order to run the program, you should run the file \textbf{main.py}. In order to change the path that is desired to be checked for being block-free: change the input list in the 7-th line in that file: \begin{verbatim}
    ui.path_free_of_blockages(<input>))
\end{verbatim}
or just use the querying user-interface system.
\end{document}


